\section{Installation}

\subsection{\svn}

The directions here are rather straight forward. Obtain tortoise from http://tortoisesvn.net/downloads. After downloading the installer one may simply run it. You do have to reboot before you see any changes to your explorer menus however.

\subsection{\git}

Firstly decide if you'd prefer \msys\ or \cyg\ as the back end. \cyg\ was the way to go historically, but it seems \msys\ is now preferred. I only found this out after installing \cyg\ and therefore describe the \cyg\ steps as far as I went before giving up and installing \msys. If you managed to install and configure \cyg\ please update this document or mail me (Carelvdam@gmail.com). Apparenty one should install either \cyg\ or \msys but not bothas some or other clshes may occur.

\subsubsection{\cyg\ Installation - Incomplete}

Download \cyg\ from http://www.cygwin.com/ and run the setup program. The link is a small icon appearing in the upper right corner of the web page. Run \verb|setup.exe| and while clicking next on each page select the following options : 
\begin{enumerate}
\item Choose a Download source : Install from internet
\item Select Root install Directory : \verb|C:\Cygwin| (or other appropriate directory) and select all users
\item Local Package Directory : \verb|D:\CygwinPackages|
\item Select your internet connection : Direct connection
\item Choose a download site : Use the default or select some other repository
\item Select packages : Go down to the net category and click the plus then add openssh (The cross should appear in the bin? column not the src? column)
\item If you want desktop icons click the check boxes and then click finnish
\end{enumerate}

Proceed to download and install \git\ from http://code.google.com/p/tortoisegit/downloads/list. When installing tortoise select the openssh option for everything else you can happily click next.

\subsubsection{\msys\ Installation}

Download \msys from http://code.google.com/p/msysgit/downloads/list selecting the option with the text "Full installer for official Git 1.7.0.2" under the summary and labels column. Run this installer  making sure to select the following
\begin{enumerate}
\item Select components : You may select "Git Bash here" but that feature is covered by tortoise, so I wouldn't.
\item Adjusting your PATH environment : Run Git from the Windows Command Prompt
\item Configuring your line ending conversions : Use Windows style line endings
\end{enumerate}

Proceed to download and install \git\ from http://code.google.com/p/tortoisegit/downloads/list. When installing tortoise select the Plinker/Putty option for everything else you can happily click next. 

\subsubsection{Configuration}
To configure \git\ right click on a folder in explorer and select Toroisegit>settings. Select General and specify the path of git, in \cyg\ this should be \verb|c:\cygwin\bin|, while in \msys this should be \verb|c:\program files\git\bin|. Click check now to confirm it's correct. Now select Git>Config and enter in your \gh\ details and select Autocrlf, safecrlf and Save as global.

To configure a \git\ repository on your drive right click on the repository folder and select \git >settings. Now select Git>remote and fill in the required details. This creates  reference that \git uses internally to manage you repositories.

\subsection{\hg}

The directions here are rather straight forward. Obtain tortoise from http://tortoisesvn.net/downloads. After downloading the installer one may simply runs it and clicks next until you are done.
